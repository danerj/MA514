\documentclass[11pt]{article}

\usepackage{amssymb}
\usepackage{amsmath}
\usepackage{amsthm}
\usepackage{indentfirst}

\title{MA 514 Homework 2}
\author{Dane Johnson}

\begin{document}
\maketitle

\section*{Exercise 4.1}

(a) Determine the SVD of the matrix:

$$A = \begin{pmatrix} 3&0\\0&- 2\end{pmatrix} \;.$$

We want a decomposition of the form $A = U\Sigma V^*$, where $U$ and $V$ are unitary matrices and $\Sigma$ a diagonal matrix containing the square roots of the singular values of $A$. We then know that

$$AA^* = U\Sigma V^* V\Sigma^*U^* = U\Sigma\Sigma^*U^* \quad A^*A = V\Sigma^*U^*U \Sigma V^* = V\Sigma^* \Sigma V^* \;.$$

In particular, $U\Sigma \Sigma^* U^*$ and $V\Sigma^* \Sigma V^*$ are a diagonalizations of $A^*A$ (note that  $\Sigma \Sigma^*$ and $\Sigma^* \Sigma$ are diagonal matrices. Therefore we can determine $\Sigma$ and $V$ by determining the eigenvalues of $A^*A$ and finding corresponding eigenvectors.

$$A^*A = \begin{pmatrix} 9 & 0 \\ 0 & 4\end{pmatrix}$$

Since $A^*A$ is diagonal, the eigenvalues are the diagonal elements. The eigenvalues of $A^*A$ are the squares of the singular values of $A$. Thus, $\sigma_1^2 = 9, \sigma_2^2 = 4$ and next we find corresponding normalized eigenvectors to use a columns of $V$ (so that $V$ will be unitary). In this case, since $A^*A$ is diagonal, we can find eigenvectors by inspection as $v_1 = e_1$ and $v_2 = e_2$. So we have 

$$\Sigma = \begin{pmatrix} \sqrt{9} &0 \\ 0 & \sqrt{4}\end{pmatrix} = \begin{pmatrix} 3 &0 \\ 0 & 2 \end{pmatrix} \quad V = \begin{pmatrix}
1&0 \\ 0 &1 \end{pmatrix}$$

Next, use the requirement that $AV = U\Sigma$ to find $U$.

$$U\Sigma = AV = A = \begin{pmatrix}
3 &0 \\ 0 &-2 \end{pmatrix}$$

Since $\Sigma$ scales the first column of $U$ by $\sqrt{\sigma_1} = 3$ and scales the second column by $\sqrt{\sigma_2} = 2$, so we have

$$u_1 = \begin{pmatrix} 1 \\ 0 \end{pmatrix} \;, u_2 = \begin{pmatrix}
0 \\ -1 \end{pmatrix} \implies U  = \begin{pmatrix}
1 & 0 \\ 0 & -1 \end{pmatrix} \;.$$

We conclude that an SVD factorization of $A$ is given by
$$A = U\Sigma V^* = \begin{pmatrix}
1 & 0 \\ 0 & -1 \end{pmatrix} \begin{pmatrix}
3 & 0 \\ 0 & 2 \end{pmatrix} \begin{pmatrix}
1 & 0 \\ 0 & 1 \end{pmatrix}$$

\section*{Exercise 4.2}

Suppose $A$ is an $m\times n$ matrix and $B$ is the $n\times m$ matrix obtained by rotating $A$ ninety degree clockwise on paper. We prove that $A$ and $B$ have the same singular values. \\

Let $A$ be written as

$$A = \begin{pmatrix}
a_{11} & a_{12} & ... & a_{1n} \\
a_{21} & a_{22} & ... & a_{2n} \\
... & ... & ... & ... \\
a_{m1} & a_{m2} & ... & a_{mn} \end{pmatrix}$$

Consider $A^T$ (we do not consider $A^*$ in case this is different from $A^T$):

$$A^T = \begin{pmatrix}
a_{11} & a_{21} & ... & a_{m1} \\
a_{12} & a_{22} & ... & a_{m2} \\
... & ... & ... & ... \\
a_{1n} & a_{2n} & ... & a_{mn} \end{pmatrix}$$

We can reverse the ordering of the columns of $A^T$ by multiplying by the $m\times m$ matrix $P$ that has 1s on the antidiagonal and 0s elsewhere:

$$A^TP = \begin{pmatrix}
a_{11} & a_{21} & ... & a_{m1} \\
a_{12} & a_{22} & ... & a_{m2} \\
... & ... & ... & ... \\
a_{1n} & a_{2n} & ... & a_{mn} \end{pmatrix}
\begin{pmatrix}
0 &... & 0& 1\\
0& ... &1 & 0\\
... & ... & ... & ...\\
1 &0 & ... & 0
\end{pmatrix} = 
\begin{pmatrix}
a_{m1} & ... & a_{21} & a_{11} \\
a_{m2} & ... & a_{22} & a_{12} \\
... & ... & ... & ...\\
a_{mn} & ... & a_{2n} & a_{1n}
\end{pmatrix} = B
$$

Therefore, we have found the matrix equation that relates the matrix $A$ and $B$ from the given description of how $B$ is obtained from $A$. Next, since $A$ has an SVD $A = U\Sigma V^*$, we see that $A^T = (V^*)^T\Sigma^T U^T = (V^*)^T\Sigma U^T$. This shows that $A$ has the same singular values as $A^T$ (even if the left and right singular vectors may change). Also, since $P$ is an orthogonal matrix, we have $(A^TP)(A^TP)^* = A^TPP^*(A^T)^* = A^T(A^T)^*$. This implies that the singular values of $A^TP$ are the same as $A^T$ (since these are the square roots of the eigenvalues of $A^T(A^T)^* = (A^TP)(A^TP)^*$. But since $A^TP = B$, we have shown that the singular values of $A^TP=B$ are the same as $A$. Thus, the singular values of $A$ are the same as the singular values of $B$, as desired. 

\section*{Exercise 4.3}

See the MATLAB script for this exercise.

\section*{Exercise 5.3}

Consider the matrix 
$$A = \begin{pmatrix}
-2 & 11 \\ -10 & 5
\end{pmatrix} \; .$$

(a) We will determine a real SVD of $A$ of the form $A = U\Sigma V^T$ such that one has the minimal number of minus signs in $U$ and $V$. Using $AA^* = AA^T = U\Sigma\Sigma^*U^* = U\Sigma^2U^T$, we have that $U\Sigma U^T$ is a diagonalization of $AA^T$. So we will find the eigenvalues, $\sigma_1^2$ and $\sigma_2^2$ of $\Sigma^2$ and corresponding normalized eigenvectors to form $U$. 

\begin{align*}
0 &= \text{det}(AA^T - \sigma^2 I) \\
&= \text{det} \left(\begin{pmatrix}
125 - \sigma & 75 \\ 75 & 125-\sigma^2
\end{pmatrix} \right) \\
&= (125-\sigma^2)^2 - 75^2\\
& \implies \sigma_1^2 = 200, \quad \sigma_2^2 = 50
\end{align*}

Next find an eigenvector for $\sigma_1^2 = 200$.

$$\begin{pmatrix}
125-200 & 75 & 0\\ 75 & 125-200 & 0
\end{pmatrix} = \begin{pmatrix}
-75 & 75 & 0\\ 75 & -75 & 0
\end{pmatrix} \sim  \begin{pmatrix}
1&-1&0 \\ 0&0&0
\end{pmatrix}$$

$$\implies u_1 = \begin{pmatrix}
\frac{1}{\sqrt{2}} \\ \frac{1}{\sqrt{2}}
\end{pmatrix}$$


Next find an eigenvector for $\sigma_2^2 = 50$.

$$\begin{pmatrix}
125-50 & 75 & 0\\ 75 & 125-50 & 0
\end{pmatrix} = \begin{pmatrix}
75 & 75 & 0\\ 75 & 75 & 0
\end{pmatrix} \sim  \begin{pmatrix}
1& 1&0 \\ 0&0&0
\end{pmatrix}$$

$$\implies u_2 = \begin{pmatrix}
\frac{1}{\sqrt{2}} \\ -\frac{1}{\sqrt{2}}
\end{pmatrix}$$
 
We take $U = \begin{pmatrix}
\frac{1}{\sqrt{2}} & \frac{1}{\sqrt{2}} \\ \frac{1}{\sqrt{2}} & -\frac{1}{\sqrt{2}}
\end{pmatrix}$ and $\Sigma = \begin{pmatrix}
\sqrt{200} & 0 \\ 0 & \sqrt{50}
\end{pmatrix} = \begin{pmatrix}
10\sqrt{2} & 0 \\ 0 & 5\sqrt{2}
\end{pmatrix}$. 

Then since $A = U\Sigma V^T$, $U^TA = \Sigma V^T$ so that $A^TU\Sigma^{-1} = V$. We can use this to find $V$:

\begin{align*}
V &= A^TU\Sigma^{-1} \\
&= \begin{pmatrix}
-2 & -10 \\ 11 & 5
\end{pmatrix} \begin{pmatrix}
\frac{1}{\sqrt{2}} & \frac{1}{\sqrt{2}} \\ \frac{1}{\sqrt{2}} & -\frac{1}{\sqrt{2}}
\end{pmatrix} \begin{pmatrix}
\frac{1}{10\sqrt{2}} & 0 \\ 0 & \frac{1}{5\sqrt{2}}
\end{pmatrix}\\
&= \begin{pmatrix}
-2 & -10 \\ 11 & 5
\end{pmatrix} \begin{pmatrix}
\frac{1}{20} & \frac{1}{10} \\ \frac{1}{20} & -\frac{1}{10}
\end{pmatrix} \\
&= \begin{pmatrix}
-12/20 & 8/10 \\ 16/20 & 6/10
\end{pmatrix}\\
&= \begin{pmatrix}
-3/5 & 4/5 \\ 4/5 & 3/5
\end{pmatrix} \;.
\end{align*} 

Therefore we have a factorization $A = U\Sigma V^T$ with:

$$U = \begin{pmatrix}
\frac{1}{\sqrt{2}} & \frac{1}{\sqrt{2}} \\ \frac{1}{\sqrt{2}} & -\frac{1}{\sqrt{2}}
\end{pmatrix}, \quad \Sigma = \begin{pmatrix}
10\sqrt{2} & 0 \\ 0 & 5\sqrt{2}
\end{pmatrix}, \quad V = \begin{pmatrix}
-3/5 & 4/5 \\ 4/5 & 3/5
\end{pmatrix} \;.$$

(b) The singular values of $A$ are $\sigma_1 = 10\sqrt{2}$ and $\sigma_2 = 5\sqrt{2}$.\\

The left singular vectors of $A$ are:

$$u_1 = \begin{pmatrix}
\frac{1}{\sqrt{2}} \\ \frac{1}{\sqrt{2}}
\end{pmatrix}, \quad u_2 = \begin{pmatrix}
\frac{1}{\sqrt{2}} \\ -\frac{1}{\sqrt{2}}
\end{pmatrix} \;.$$

The right singular vectors of $A$ are:

$$v_1 = \begin{pmatrix}
-3/5 \\ 4/5 
\end{pmatrix}, \quad v_2 = \begin{pmatrix}
4/5 \\ 3/5 
\end{pmatrix} \; .$$
\end{document}